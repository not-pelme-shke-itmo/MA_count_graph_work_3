\section{Задание 2. Исследование функции двух переменных}

\textbf{Условие.}

\begin{enumerate}[label=\Alph*.]
    \item Изобразите поверхность, заданную уравнением $z=z(x,y)$, в программе Geogebra 3D.

    Выполните следующие этапы исследования:

    \begin{enumerate}[label=\arabic*.]
        \item Найдите область определения $z = z(x, y)$.

        \item Постройте в программе Geogebra Classic (на одном листе!) семейство линий уровня $z(x, y) = c$.
        Для построения выберите 3–4 значения $c$.
        Определите тип построенных кривых (найдите уравнения линий уровня при выбранных значениях $c$).
        Если разным $c$ соответствуют кривые разных типов (например: прямые, окружности, точка), изобразите все типы линий уровня.

        \item Выберите на поверхности какую-либо обыкновенную и не стационарную точку $M_0$ (определите ее координаты $x_0, y_0, z = z(x_0, y_0)$).
        Докажите (по определению), что выбранная точка не является особой и стационарной.

        \item Найдите вектор $\overrightarrow{m}$ - направление наискорейшего спуска (подъема) в точке $M_0$.

        \item Изобразите в программе Geogebra Classic линию уровня $z = z(x_0, y_0)$ и направление.
        Проверьте их ортогональность.

    \end{enumerate}

    \item Найдите наибольшее и наименьшее значения функции $u = u(x, y)$ в области $D$:

    \begin{enumerate}[label=\arabic*.]
        \item Найдите стационарные точки внутри области.
        \item Определите, являются ли стационарные точки точками экстремума.
        \item Исследуйте значения функции вдоль границ области.
        \item Определите точки области, в которых достигаются наибольшее и наименьшее значения функции, и сами значения
    \end{enumerate}

    \vspace{5mm}

    \begin{tabular}{ccc}
        Функция $z = z(x, y)$                     & Функция $u = u(x, y)$         & Область $D$                          \\
        $\displaystyle z = \frac{8y}{x^2 + 4y^2}$ & $u = x^2 + 2x + y^2 - 4y + 4$ & $0 \leq x \leq 2, \ 0 \leq y \leq 3$
    \end{tabular}
\end{enumerate}

\vspace{10mm}
\textbf{Решение.}

It is empty but you can fill it!

\textit{Ответ}: It is empty but you can fill it!
\clearpage
